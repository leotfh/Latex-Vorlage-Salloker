\documentclass[11pt,a4paper, top=3cm, left=2.5cm, right=2.5cm, bottom=4cm]{article}
\usepackage{fontspec}
\usepackage{textcomp}
\usepackage{unicode-math}
%\setmainfont{Libertinus Serif}
\setmathfont{Libertinus Math}
% \setmainfont{XITS}
% \setmathfont{XITS Math}
% \setmainfont[ExternalLocation=Fonts/Caladea/]{Caladea-Regular.ttf}%
%   [Ligatures=TeX, % Tex
%   BoldFont=Caladea-Bold.ttf,
%   ItalicFont=Caladea-Italic.ttf,
%   BoldItalicFont=Caladea-BoldItalic.ttf]
\setmainfont[ExternalLocation=Fonts/cambria/]{cambria.ttc}%
  [BoldFont=CAMBRIAB.ttf,
  ItalicFont=CAMBRIAI.ttf]
\usepackage{chngcntr}
\usepackage{setspace}

% Set the line spacing to 19pt
\setstretch{1.37}

\usepackage{geometry}
 \geometry{
 left=25mm,
 right=25mm,
 top=30mm,
 bottom=40mm
 }
\usepackage{floatrow}
\floatplacement{figure}{H} % !ht
\floatplacement{table}{H} % !ht
\usepackage[singlelinecheck=off]{caption}
\usepackage{tabularx, booktabs}
% \DeclareCaptionFormat{custom}
% {%
%     \textbf{\small #1#2}\small #3   % Formatierung für Beschriftungen
% }
\captionsetup[figure]{justification=centering, , labelfont=bf}            % Formatierung für Beschriftungen
\captionsetup[table]{justification=raggedright, indention=.5cm, labelfont=bf}
\usepackage{matlab-prettifier}
\usepackage{minted}                     % Package für Code-Darstellung
\usemintedstyle{pastie}
% \usepackage[framemethod=tikz]{mdframed}
% \mdfdefinestyle{matlab}{%           % Environment für MatLab-Darstellung von Code
%   outerlinewidth=.5pt,
%   linecolor=gray!20!white,
%   roundcorner=2pt,
%   innertopmargin=.5\baselineskip,
%   innerbottommargin=.5\baselineskip,
%   innerleftmargin=1em,
%   backgroundcolor=gray!10!white
% }
% \newenvironment{matlabcode}{\Verbatim}{\endVerbatim}
% \surroundwithmdframed[style=matlabcode]{matlabcode}

% \definecolor{output}{gray}{0.4}
% \newenvironment{matlaboutput}{%         % Environment für MatLab-Dartstellung von Ausgabewerten
%   \Verbatim[xleftmargin=1.25em, formatcom=\color{output}]%
% }{\endVerbatim}
\usepackage[ngerman]{babel}
\usepackage{graphicx}
\usepackage{float}
\floatstyle{plaintop}
\restylefloat{table}
\usepackage{adjustbox}
\usepackage{amsmath}
\usepackage{siunitx} % Für SI Einheiten
\sisetup{locale = DE}
\usepackage{fancyvrb}
\usepackage{fancyhdr}
\usepackage{titlesec}
\usepackage{xcolor}
\usepackage{listings}
\usepackage{datetime}
\renewcommand{\thefigure}{\arabic{section}.\arabic{figure}}
\counterwithin{figure}{section}
\usepackage{multirow}
\renewcommand\thetable{\thesection.\arabic{table}}
\counterwithin{table}{section}
\usepackage[hidelinks]{hyperref}
\hypersetup{
linktoc=all
}

\makeatletter
\renewcommand\listoftables{%
        \@starttoc{lot}%
}
\renewcommand\listoffigures{%
        \@starttoc{lof}%
}
\makeatother

\lstdefinestyle{DOS}                % Ausgabeformat für Consolen-Output
{
    backgroundcolor=\color{white},
    basicstyle=\scriptsize\color{black}\ttfamily
}

\addto\captionsngerman{% Replace "english" with the language you use
  \renewcommand{\contentsname}{Inhalt}}

\usepackage[utf8]{inputenc}
% \usepackage[T1]{fontenc}
% \usepackage{lmodern}
\usepackage{color}
% \usepackage{hyperref}
\usepackage{amsfonts}
\usepackage{epstopdf}
\usepackage[table]{xcolor}
\usepackage{matlab}